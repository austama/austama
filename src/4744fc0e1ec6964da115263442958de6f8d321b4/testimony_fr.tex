Lors de ma première vaccination avec Pfizer, j'ai eu des symptômes légers, comme
des palpitations, mais pas du tout graves. Mais j'ai reçu ma deuxième injection
le 8 septembre, et je n'oublierai jamais ce jour-là. Le 10 septembre, j'ai
consulté mon médecin traitant pour des problèmes cardiaques. Il a fait un
électrocardiogramme et a constaté un petit problème. Le soir même, j'étais aux
urgences. Ils ont fait trois électrocardiogrammes et ont pensé que je faisais
une crise cardiaque.

Ils ont fait un cathétérisme cardiaque, mais ont dit que mes artères étaient en
bon état. La douleur était si intense que j'ai commencé à avoir des bouffées de
chaleur et à m'évanouir. Ils m'ont alors mise aux soins intensifs et m'ont dit
que j'avais une péricardite. Je suis rentré chez moi ! Quatre jours plus tard,
c'était pire. Ils ont fait un autre cathétérisme cardiaque et m'ont dit que les
artères étaient de nouveau en bon état, mais cette fois ils ont dit que j'avais
une myocardite et une inflammation des poumons !

À l'hôpital, je faisais des traitements respiratoires toutes les six
heures. J'ai pris beaucoup de médicaments pendant un mois. J'ai pris des
stéroïdes, des anti-inflammatoires et des médicaments contre la douleur.

J'ai encore des poussées. Je continue à faire des allers-retours chez le
médecin. Aujourd'hui, j'ai encore des problèmes avec la douleur qui réagit à mon
cœur si je fais trop d'efforts. Alors maintenant, je dois y aller doucement.

J'aimerais que tout le monde sache qu'il existe un risque de péricardite ou de
myocardite si l'on prend le vaccin. C'est vraiment un événement qui change la
vie, juste pour dire que je suis vacciné !
