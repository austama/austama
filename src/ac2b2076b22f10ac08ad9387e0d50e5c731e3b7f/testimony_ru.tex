История Келли:

Келли упорно борется с синдромом Гиллана-Барре (СГБ), редким аутоиммунным
заболеванием, которое поражает нервы и вызывает паралич.

Она может двигать только руками, шеей и пожимать плечами. Она не может брать и
удерживать предметы. Келли не может есть самостоятельно уже четыре недели из-за
того, что её глотание было поражено заболеванием, и полностью зависит от зонда
для кормления.

Келли борется с болезнью уже более 7 недель и впервые попала в больницу 30 марта
2021 года. Её врачи считают, что болезнь была вызвана второй вакциной Pfizer.

Синдром Гиллана-Барре Келли безжалостен и не поддается лечению. Это очень
отягчает ситуацию, так как врачи не нашли, какие методы лечения или комбинации
методов лечения необходимы, чтобы её болезнь достигла стабильного плато
(перестала прогрессировать и рецидивировать). Врачи считают, что этот случай СГБ
имеет тенденцию к хронической форме, называемой хронической воспалительной
демиелинизирующей полинейропатией (ХВДП).

За последние семь недель она перенесла следующие очень тяжелые процедуры: 7 раз
иммуноглобулины (ВВИГ), 10 раз плазмаферез (сокращенно Plex) и высокодозную
химиотерапию внутривенно (5/5).

Её голос был очень слабым в течение нескольких недель. Ей помогает дыхательный
аппарат Bi-pap, который помогает ей избежать искусственной вентиляции легких в
течение нескольких недель, пока её диафрагма не станет слишком слабой, не говоря
уже о процедурах введения трубки для еды на длительный срок, что вынуждает
постоянно использовать вентилятор благодаря трахеостомии.

Келли была и остается замечательной медсестрой, дочерью, сестрой, тетей,
племянницей и замечательным другом. Ей нравилось заниматься йогой, бегать,
ходить в походы и путешествовать. Недавно, помимо того, что она работала
медсестрой, Келли шла к получению степени магистра медсестер в Университете
Гранд-Каньона и даже получила стипендию Dawn Gross от онкологического центра
Андерсона.

Келли была здорова до вакцины Pfizer и получил её якобы для защиты своих
пациентов.
