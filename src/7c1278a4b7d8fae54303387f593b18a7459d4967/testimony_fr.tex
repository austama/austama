Jennifer, 37 ans, mère de deux enfants, sans problèmes de santé, a soutenu le
programme de vaccination et souhaitait recevoir sa deuxième dose, mais elle a
reçu des conseils contradictoires de la part des médecins après avoir souffert
d'acouphènes (bourdonnements d'oreilles) et d'hyperacousie (sensibilité aux
sons) chroniques et débilitants depuis qu'elle a reçu sa première dose du vaccin
Pfizer début juillet 2021.

Elle craint que la deuxième dose n'aggrave son état. Son médecin généraliste et
un consultant en otorhino-laryngologie qu'elle a consultés lui ont tous deux dit
que cela pourrait bien se produire.

Jennifer dit qu'elle a parlé à sept autres médecins, consultants et autres
professionnels de la santé et que tous, à l'exception d'un seul, lui ont dit
qu'ils ne pouvaient pas lui conseiller de prendre ou non la deuxième dose ou lui
ont conseillé de ne pas le faire.

Ses médecins l'ont informée que son état pourrait désormais être permanent.

Elle a écrit au ministre de la Santé, Stephen Donnelly, pour lui demander de
d'offrir de meilleurs soins aux personnes dans sa situation.

Elle dit avoir reçu un manque de soutien de la part du HSE et avoir été “laissée
dans l'incertitude”, souhaitant recevoir la seconde dose de vaccin pour sa
sécurité et celle de sa famille - et craignant de n'avoir reçu qu'une seule dose
- mais elle s'inquiéte du risque d'exacerber le niveau de son acouphène si elle
reçoit le second vaccin.

“Je suis une fervente partisane de la vaccination et je suis chaque jour
reconnaissante des bienfaits des progrès de la médecine”, a-t-elle déclaré, mais
elle a le sentiment d'avoir été “confrontée à la roulette russe” de faire face
au risque de contracter le Covid, qu'elle ne sous-estime pas, ou au risque
d'aggraver son état.

