Tayla raconte son histoire le 05/11/21 :

Le 25 août 2021, j'ai reçu le vaccin Pfizer Covid-19 pour pouvoir garder mon
emploi d'infirmière.

Deux semaines plus tard, mon rythme cardiaque au repos est passé de 60 à 100
bpm. Deux semaines plus tard, j'ai eu une grave infection des amygdales de grade
3-4 (pus, ulcères et saignement des amygdales). L'infection s'est déclarée très
rapidement, en l'espace de trois heures.

Lors de ma première visite à l'hôpital, j'ai été renvoyé chez moi avec des
antibiotiques et on m'a dit que mon rythme cardiaque devrait revenir à la
normale dans les 48 heures. Au bout de 48 heures, mon rythme cardiaque était
toujours élevé, alors nous nous sommes rendus dans un autre hôpital, où l'on m'a
dit que j'étais déshydratée, alors on m'a hydratée par voie orale, on m'a donné
un stéroïde à action rapide et on m'a renvoyée chez moi en me disant : ``Votre
rythme cardiaque va redescendre dans quelques jours''. On m'a même dit qu'il
était normal pour une jeune fille de 22 ans d'avoir un rythme cardiaque au repos
de 110, même si je savais que ce n'était pas normal pour moi.

Deux jours plus tard, j'ai pris rendez-vous avec mon médecin, qui m'a fait un
électrocardiogramme et m'a posé un moniteur cardiaque pendant 24 heures. Le
moniteur a montré que ma fréquence cardiaque au repos était d'environ
110-130. Elle m'a envoyé une recommandation pour faire une écho, que je n'ai pu
obtenir qu'une semaine plus tard. Pendant cette période d'attente, j'avais
terminé mes antibiotiques, mais mes piercings d'oreille, que j'avais depuis un
an, ont commencé à produire du pus. Un autre voyage à l'hôpital où j'ai attendu
dans la salle d'attente avec 5 autres personnes qui avaient une oppression
thoracique, un essoufflement et un rythme cardiaque élevé - ils avaient tous à
peu près mon âge, entre 20 et 30 ans, et venaient tous d'avoir leur vaccin
Covid-19 !

J'ai attendu de 19h30 à 2h du matin pour voir un médecin. Remarquez que pendant
tout ce temps, j'ai toujours des symptômes d'oppression thoracique,
d'essoufflement, de tension artérielle élevée, de rythme cardiaque élevé et de
vertiges lorsque je me tiens debout. Lorsque j'ai vu un médecin, il m'a fait
passer une radiographie de la poitrine et m'a fait prendre des analyses de sang,
mais il est arrivé à la conclusion que tout cela n'était que de l'anxiété. Ils
m'ont donné un diazépam et m'ont renvoyé chez moi. C'était le 5 octobre et nous
sommes maintenant le 6 novembre.

Après avoir été écarté et mis de côté par de nombreux médecins et spécialistes,
ils ne trouvent toujours pas de diagnostic pour moi. Un médecin m'a même dit
qu'il n'y avait plus de tests à faire, alors que je n'avais subi que quelques
tests sanguins, une échographie et un moniteur de halter.

Je n'ai pas travaillé depuis 6 semaines maintenant et j'ai reçu le vaccin
Covid-19 pour ne pas perdre mon emploi, mais maintenant je suis ici sans
travail, sans argent, à cause des effets secondaires du vaccin. Je suis toujours
frustrée de voir que personne ne croit qu'il y a des gens qui ont des réactions
indésirables au vaccin. On ne m'a encore rien diagnostiqué, mais je souffre
toujours d'oppression thoracique, d'essoufflement et d'une fréquence cardiaque
au repos de 110. Avant le vaccin, j'allais au gymnase 5 fois par semaine et je
maintenais un état cardiaque sain. Depuis, je ne suis plus moi-même. Ma santé
mentale a considérablement diminué et ma confiance dans le système de soins de
santé aussi.

Lorsqu'une procédure médicale comporte un risque, il devrait toujours être
possible de choisir, quoi qu'il arrive. La seule raison pour laquelle j'écris
ces lignes aujourd'hui est de sensibiliser davantage les gens et d'espérer que
personne d'autre n'aura à subir ce que je vis.

Pour tous ceux qui vivent la même chose que moi, vous n'êtes pas seuls, car il y
a des centaines ou des milliers d'autres personnes exactement comme nous. En
soins infirmiers, on nous parle toujours de la qualité de vie d'une personne et,
à l'heure actuelle, je peux dire que je n'ai aucune qualité de vie en essayant
de vivre comme ça.

Je ne serai plus réduite au silence.

