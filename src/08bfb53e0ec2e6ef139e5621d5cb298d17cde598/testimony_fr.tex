Je m'appelle Jessica. J'ai 33 ans, je suis une épouse, une mère et une
infirmière.

Je n'ai pas été vaccinée jusqu'à ce que mon entreprise tombe sous le coup de la
loi. Le 12 octobre, j'ai donc reçu ma première dose du vaccin Moderna. Certains
des symptômes que j'ai ressentis ne sont apparus qu'au bout de deux semaines, il
s'agissait de bourdonnements d'oreilles, d'un goût métallique dans la bouche,
d'une inflammation de l'œil droit, d'une faiblesse, d'un tintement ou d'une
sensation de faiblesse du côté droit et de migraines.

Je suis allée voir mon PCP le 28 octobre et j'ai reçu des médicaments pour les
migraines ainsi que des stéroïdes. Le 9 novembre, j'ai décidé de passer au
second vaccin. J'ai deux enfants et un mari, qui sont tout pour moi.

Mon médecin pensait, et pense toujours, que tout est lié au vaccin
Moderna. Cependant, ce dernier vaccin a fait des ravages dans mon corps. Ma
tension artérielle est montée en flèche (je me suis retrouvée aux urgences avec
une tension de type AVC) le 11 novembre 2021. Le diagnostic que j'ai reçu était
“HTN transitoire, réaction indésirable à la vaccination”.

Donc maintenant, j'ai deux prestataires, mon PCP et même le pharmacien de la
pharmacie où j'ai reçu mes vaccins, qui me disent tous que c'est lié à 100\% au
vaccin Moderna.

Après cette visite aux urgences, j'ai passé une semaine clouée au lit avec des
migraines débilitantes. Ma tension et ma fréquence cardiaque étaient toutes deux
extrêmement élevées. Avant cela, ma tension “normale” était de 120/70 et ma
fréquence cardiaque de 70-80. Maintenant, ma tension systolique varie entre 130
et 190 et ma tension diastolique entre 80 et 120. Ma fréquence cardiaque peut
être de 60 pendant une seconde et, la seconde suivante, de 140. Mon brassard de
tension artérielle est muni d'un symbole qui indique que mon cœur n'est pas en
rythme (ce symbole n'apparaît que lorsque je me sens “bizarre” ou que j'ai une
légère douleur à la poitrine). J'ai des douleurs thoraciques, une sensation
d'oppression, un essoufflement, un brouillard cérébral EXTRÊME, de la fatigue
et, honnêtement, de la fatigue en général.

Je veux juste être capable d'être une épouse et une maman et je ne suis pas
capable d'être ces choses pour ma famille. Je travaille à la demi-journée
jusqu'à ce que je puisse régler ce problème. Qu'elle ironie… prendre l'injection
pour garder mon emploi, et pourtant je suis là, incapable physiquement de tenir
plus de 4-6 heures par jour. Avant cela, je faisais des journées de 10 à 12
heures.

Mardi soir, alors que je regardais un film avec mon mari en bas (j'étais
allongée sur le dos), je lui ai dit que je me sentais “bizarre” et que je devais
m'asseoir, car j'avais l'impression de m'évanouir lentement en étant
allongée. Mon rythme cardiaque était de 140 et j'ai eu tellement peur que je
n'ai même pas essayé de monter les escaliers pour aller me coucher. J'ai
simplement dormi en bas. Je me souviens qu'avant le vaccin, lorsque je faisais
de l'exercice, mon rythme cardiaque ne pouvait jamais dépasser 120, quoi que je
fasse ! Mais maintenant, tout ce que j'ai à faire, c'est de rester assise,
complètement immobile. Pendant ces épisodes, j'ai l'impression que je vais
m'évanouir, mais je ne l'ai pas encore fait - je croise les doigts. Je surveille
maintenant ma tension artérielle plusieurs fois par jour, j'ai commencé à
prendre du Metoprolol tous les jours pour l'hypertension et le rythme cardiaque,
j'ai un rendez-vous chez le cardiologue mardi prochain (pour un nouvel ECG, une
écho, une épreuve d'effort et un moniteur holter) et un suivi après ce
rendez-vous avec mon médecin traitant.

Je suis vraiment désolée pour tous ceux qui traversent cette épreuve. Je suis
vraiment désolée d'avoir fait ce que je pensais être le mieux pour ma
famille. Avec le recul, je n'aurais jamais pris la première dose. Un rapport
VAERS a été déposé, mais je ne l'ai pas encore vu. Je ne suis pas folle. Vous
n'êtes pas fou. Ce que nous vivons est très réel et je n'arrêterai pas tant que
quelqu'un ne nous entendra pas. Si un patient est allergique et/ou a une
réaction indésirable à un antibiotique, est-il traité de fou ? On lui fait honte
et on lui dit que “tout est dans sa tête” ? Non. On les croit et on les écoute…
nous le ferons aussi !

Mise à jour… aujourd'hui, c'est le 10e jour après la 2e piqûre. La nuit
dernière, j'ai passé une autre nuit aux urgences car je pensais que j'avais une
crise cardiaque et que je devais être évaluée. L'ECG a montré cette fois une
bradycardie sinusale. D-dimère négatif. CXR négatif. Toutes les enzymes
cardiaques sont dans les limites normales. J'ai reçu une injection de Torodol
pour la douleur thoracique, sans soulagement. La pression sanguine semblait
correcte. La saturation en oxygène chute dans les 80 \% mais remonte
immédiatement aux limites de la normale sans oxygène. L'urgentiste, le CINQUIÈME
fournisseur de soins médicaux, a reconnu que c'était à 1000\% lié au vaccin et
que ça devrait juste “s'estomper”. Mais combien de temps ? Combien de temps
faut-il pour que cela “s'estompe” ?

Je me bats mentalement, physiquement et FINANCIÈREMENT. Regardez les factures
médicales que j'accumule, et cela ne fait qu'un peu plus d'un mois que cela dure
!

Mais je ne m'arrêterai pas. Je ne m'arrêterai pas tant que je n'aurai pas de
réponses. Jusqu'à ce que j'aie un traitement qui me permette de vivre une vie
semi-normale.

Mon mari et moi discutions en rentrant à la maison hier soir et, au cours des
DOUZE années que nous avons passées ensemble, nous avons finalement décidé que
je n'avais été évaluée aux urgences qu'UNE SEULE fois, et c'était à cause d'une
fausse couche après un accident de voiture. Je ne cours pas chez le médecin
chaque fois que j'éternue. En fait, en tant qu'infirmière, nous sommes les pires
patients, et encore plus pour les urgences. J'ai été évalué par des médecins
plus souvent au cours du dernier mois et demi qu'au cours des deux dernières
années. Cela s'arrêtera-t-il un jour ?
