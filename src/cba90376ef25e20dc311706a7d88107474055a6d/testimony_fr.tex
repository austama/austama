Ma fille de 39 ans, Charlene, a malheureusement eu une réaction indésirable au
vaccin Pfizer anti-covid.

Charlene a reçu son deuxième vaccin anti-covid le 11 juin, puis, le 13 juin,
elle a été transportée d'urgence à l'hôpital au milieu de la nuit. Elle était
essoufflée, avait des douleurs cardiaques, le bout de ses doigts et ses lèvres
étaient bleus, elle était complètement léthargique et avait 105 (fahrenheit,
soit 40 degrés celsius) de fièvre.

On lui a rapidement diagnostiqué une myocardite et une péricardite, ainsi qu'un
petit trou dans son cœur. Il s'agit d'un événement rare qui a été constaté chez
certaines personnes comme une “complication” du vaccin. Charlene souffrait
auparavant d'un problème cardiaque sous-jacent, car elle est née avec un souffle
au cœur.

Elle a été suivie par son médecin de famille, des cardiologues et une équipe de
médecine interne. L'équipe s'efforce de comprendre les problèmes et de faire de
son mieux pour l'aider à se rétablir.

Les médecins lui ont conseillé de rester en arrêt de travail, mais pour
l'instant, on ne sait pas quand elle pourra y retourner, car elle en est à son
cinquième mois d'arrêt. Il est triste que quelqu'un, comme ma fille, qui est
toujours prête à travailler pendant la pandémie, soit incapable de travailler en
raison d'une réaction indésirable au vaccin Pfizer.

Elle a toujours une vision très positive de la vie, et cela me brise le cœur de
la voir lutter dans cette période. Elle est mère célibataire de deux
adolescents, qu'elle adore profondément, et ses enfants ont été touchés par
cette crise.

