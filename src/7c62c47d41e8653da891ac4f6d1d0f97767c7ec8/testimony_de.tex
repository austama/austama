Die slowenische Regierung setzte den Impfstoff Janssen nach dem Tod der
20-jährigen Studentin Katja Jagodic aus.

Katja wurde geimpft, weil dies von den Universitäten in Slowenien vorgeschrieben
war. Sie erhielt die Impfung am Donnerstag, dem 16. September 2021. Kurz darauf
wurde sie krank und wurde am Montag, dem 27. September, in das
Universitätsklinikum (UKC) in Ljubljana eingeliefert. Ihre Symptome waren so
beäng, dass eine sofortige medizinische Behandlung erforderlich war. Sie starb
am 29. September in den frühen Morgenstunden.

Igor Rigler, Leiter der Abteilung für Notfallneurologie am Universitätsklinikum,
sagte: ``Sie war schwerkrank. Sie hatte gleichzeitig Blutgerinnsel und niedrige
Thrombozytenwerte, wodurch Blutungen in ihrem Gehirn auftraten. Wir haben alles
getan, was die moderne Medizin unter solchen Umständen tun kann. Wir haben alle
Experten konsultiert, die am UKC auf diesem Gebiet tätig sind -
Gefäßspezialisten, Hämatologen, Nephrologen und Neurologen. Wir sind die
Behandlung aggressiv angegangen, aber leider konnten wir ihr nicht helfen''.

Der slowenische Gesundheitsminister Janez Poklukar ordnete nach dieser Nachricht
die Aussetzung des Impfstoffs Covid-19 von Janssen an.

Der Vater der verstorbenen 20-Jährigen war am 29. September 2021 bei einer
Demonstration gegen den ``COVID-Pass'' dabei. Die Teilnehmer der Kundgebung
legten eine Schweigeminute ein, bevor er sagte: ``Sie wollte die Freiheit haben,
so wie alle anderen hier. Am 16. September ging sie zu Janssen, um sich impfen
zu lassen, damit sie nicht durch Unsinn eingeschränkt wird. Heute ist sie nicht
mehr da. Sie war 20 Jahre alt.''

Er fuhr fort: ``Wenn man zur Impfstelle kommt, warnt einen niemand vor möglichen
Komplikationen. Niemand warnt davor, dass die Impfung mit dem Janssen-Impfstoff
für Menschen unter 40 Jahren nicht empfohlen wird. Es gibt nur Zahlen. Es ist
nur von Prozentsätzen die Rede. Meine Katja war keine Prozentzahl. Sie war meine
Katja'', sagte der emotionale Vater.

Nach den Reden bewegte sich die Menge zum Präsidentenpalast und setzte ihren Weg
durch die Straßen von Ljubljana fort. Die Demonstranten, die gegen den
obligatorischen COVID-Pass sind, um zu arbeiten, eine Tankstelle oder ein
Einkaufszentrum zu erreichen, blockierten schließlich eine Schnellstraße. Die
Polizei wurde mit Wasserwerfern in das Gebiet geschickt, um gegen die
Demonstranten vorzugehen.
