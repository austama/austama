Правительство Словении приостанавливает действие вакцины Janssen Covid-19 после
смерти 20-летней студентки Кати Ягодич.

Катя была привита после установления обязательности вакцинации для университетов
в Словении. Она получила вакцину в четверг, 16 сентября 2021 года. Вскоре после
этого у неё появилось несколько проблем со здоровьем, и в понедельник, 27
сентября, она была госпитализирована в Университетский клинический центр (UKC) в
Любляне. Её симптомы были настолько серьёзными, что потребовалась немедленная
медицинская помощь. Она умерла 29 сентября рано утром.

Игорь Риглер, заведующий отделением неотложной неврологии Университетского
клинического центра, сказал: «Она была в очень тяжелом состоянии. У неё
одновременно были тромбы и низкие тромбоциты, что вызвало кровоизлияние в
мозг. Мы сделали всё, что может сделать современная медицина в таких
условиях. Мы проконсультировались у всех специалистов, работающих в этой области
в больничном центре – сосудистых специалистов, гематологов, нефрологов и
неврологов. Мы подошли к лечению очень активно, но, к сожалению, ничего не могли
сделать».

Министр здравоохранения Словении Янез Поклюкар приказал приостановить выпуск
вакцины Janssen Covid-19 после этой новости.

Отец погибшей 20-летней девушки присутствовал во время манифестации 29 сентября
2021 года против «COVID-Pass». Участники митинга почтили минутой молчания память
Кати прежде чем он выступил: «Она хотела иметь свободу, как и все здесь. 16
сентября поехала делать прививку в Janssen, чтобы избежать глупых
ограничений. Сегодня её больше нет. Ей было 20 лет.

И продолжает: «Когда приезжаешь на место вакцинации, тебя никто не предупреждает
о возможных осложнениях. Никто не предупреждает вас о том, что вакцинация
прививкой Janssens не рекомендуется лицам моложе 40 лет. Есть только цифры. Мы
говорим только о процентах. Моя Катя не была процентом. Она была моей Катей», —
сказал скорбящий отец.

После выступлений толпа направилась к президентскому дворцу и продолжила свой
путь по улицам Любляны. Демонстранты, которые выступают против обязательного
COVID-Pass для работы, доступа к заправочной станции или торговому центру, в
итоге перекрыли шоссе. На место демонстраций была направлена полиция с
водометами для борьбы с протестующими.
