Lorena est une infirmière diplômée depuis neuf ans et a passé l'année 2020 à
travailler en première ligne. En janvier 2021, elle a reçu le vaccin Pfizer et,
depuis sa deuxième dose, elle n'a cessé de se battre pour sa santé et cela fait
maintenant près d'un an qu'elle est incapable de travailler.

Voici son histoire, avec ses propres mots :

J'ai été confrontée à des symptômes qui m'ont valu de nombreuses visites chez le
médecin, aux urgences et à l'hôpital. Tout a commencé après avoir reçu ma
deuxième dose du vaccin Pfizer. Je souhaite partager mon expérience dans
l'espoir de faire prendre conscience de ce que certains d'entre nous ont vécu
avec ce vaccin Covid-19 et, ce faisant, de faire avancer la recherche pour nous
aider à trouver un traitement qui nous soulagera des effets secondaires dont
nous souffrons.

Le 11 janvier, j'ai reçu ma deuxième dose du vaccin Pfizer. Quinze heures plus
tard, je me suis réveillée avec des douleurs à la poitrine, de la fièvre, une
fréquence cardiaque de 160, des engourdissements, une quasi-syncope et un
essoufflement. J'ai été hospitalisée pendant 2 jours et on m'a dit que c'était
ma réponse immunitaire au vaccin. Pendant cette hospitalisation, j'ai subi des
analyses, des scanners, un échocardiogramme et des perfusions. Quelques analyses
étaient anormales, notamment ma DDimère et mon acide lactique. Le scanner
thoracique n'a pas révélé d'EP. J'ai été renvoyé à la maison et mon rythme
cardiaque était toujours élevé, mais pas autant. J'ai également commencé à
ressentir des tremblements internes. Je me suis reposé après mon
hospitalisation, mais le 30 janvier, cela s'est reproduit. Les mêmes symptômes
exacts. Cette fois, j'ai été hospitalisée pendant 4 jours. Mon D-Dimer était
plus élevé, j'ai donc subi d'autres examens. Ils n'étaient pas sûrs de ce qui se
passait, mais on m'a mis sous propranolol car mon rythme cardiaque ne voulait
pas baisser cette fois. Cela a fait baisser ma tension artérielle, déjà basse,
et mon cardiologue a changé le traitement en Corlanor pour les patients
externes. Depuis, j'ai eu de nombreuses visites aux urgences en raison d'une
fréquence cardiaque élevée, des épisodes de quasi-syncope, des rendez-vous avec
mon pcp, mon cardiologue, mon cardiologue EP, mon neurologue, mon
gastroentérologue, mon pneumologue et j'attends mon rendez-vous avec le
neurologue autonome de Stanford. J'ai subi de nombreux tests, notamment une
échocardiographie, un holter cardiaque de deux semaines, une épreuve d'effort,
des analyses de laboratoire, l'implantation d'un enregistreur à boucle, des
tomographies, une IRM, etc.

Depuis que tout cela a commencé, mes effets secondaires ont été et sont toujours
les suivants : tachycardie, douleurs thoraciques intermittentes, PACs/PVCs,
essoufflement, tremblements internes, spasmes musculaires/secousses,
vertiges/étourdissements constants, maux de tête, brouillard cérébral, épisodes
quasi syncopaux, perte d'appétit par intermittence (j'ai perdu plus de 30 livres
depuis janvier), fatigue, douleurs articulaires intermittentes aléatoires,
etc. Mon cardiologue à Modesto dit qu'il s'agit peut-être d'une tachycardie
sinusale inappropriée, mais je n'ai pas reçu de diagnostic officiel. Je verrai
le cardiologue du PE à Stanford en novembre. Dans le passé, j'ai eu des vertiges
et des maux de tête, mais rien d'aussi grave ni d'aussi constant que ce qui est
débilitant maintenant. J'ai reçu du COVID il y a quelques semaines et il a
définitivement aggravé mes effets secondaires.

Ma prochaine étape est de trouver un médecin fonctionnel/naturopathe pour faire
des analyses approfondies et sortir des sentiers battus, car nous ne savons
toujours pas comment soulager ces effets indésirables.

Je suis tellement fatiguée et frustrée de me sentir comme ça. Je ne me suis pas
sentie moi-même depuis le 12 janvier. Ma vie a complètement changé. Je m'ennuie
de ma vie. Je m'ennuie de me sentir moi-même. Je m'ennuie de ne pas me sentir
mal chaque jour. J'espère que ces effets secondaires que certaines personnes
ressentent feront l'objet de recherches et que des réponses seront trouvées
quant à leur raison d'être et à la façon de les traiter.

Quelques mises à jour récentes :

14 octobre 2021 - Je suis allé à Stanford et j'ai été vu par un neurologue dans
leur clinique de neuroscience du mouvement. Le neurologue que j'ai vu pense que
j'ai un POTS et/ou un trouble auto-immun d'une certaine sorte qui a été
déclenché par le vaccin. Elle a demandé d'autres analyses qui ont été faites
aujourd'hui. Elle m'a adressé à sa collègue de Stanford, qui est neurologue
autonome, pour que je fasse un test de POTS et un test EMG. Elle m'a également
orientée vers un neuro-immunologue de l'UCSF qui travaille avec des patients
ayant reçu le vaccin et présentant des effets secondaires à long terme. Elle
pense qu'il peut vraiment m'aider à comprendre certaines choses. Je vais
également commencer une thérapie vestibulaire à Stanford et une nouvelle
ordonnance a été prescrite pour m'aider à lutter contre les vertiges constants
que je subis depuis que mon traitement actuel ne m'aide plus.

15 novembre 2021 - J'ai eu une consultation avec un neurologue autonome à
Stanford. Il m'a ordonné de passer des tests autonomes pour une suspicion de
POTS en décembre, ainsi qu'un nouvel échocardiogramme. Pour l'instant, il m'a
dit d'augmenter ma dose de Corlanor, ma consommation de sel et d'eau pour
soulager ma tension artérielle basse et ma fréquence cardiaque élevée. Une fois
que nous aurons les résultats, il mettra à jour le plan de soins.

22 novembre 2021 - J'ai passé en revue quelques analyses avec un autre de mes
spécialistes et j'ai reçu des nouvelles accablantes sur ma santé. Mes analyses
ont montré que mon système immunitaire ne fonctionne pas de manière optimale,
que mon corps présente une inflammation et que j'ai plusieurs
maladies/co-infections que mon corps essaie actuellement de combattre. Je suis
atteinte de Lyme, de candida actif, d'Epstein Barr réactivé et de moisissure. Je
vais bientôt commencer un traitement pour ces maladies.

23 novembre 2021 - J'avais une douleur constante à la poitrine, une tachycardie
et un essoufflement, alors je suis allée aux urgences. Le scanner s'est révélé
négatif pour une EP.

24 novembre 2021 - J'ai vu un neurologue qui a examiné les résultats de mon IRM
cérébrale. Il n'est pas certain de la nature de la lésion qui a été
découverte. Cette lésion n'était pas présente dans mon IRM de février
2020. Heureusement, la lésion se trouve dans la zone optique latérale. Je vais
passer une nouvelle IRM avec contraste cette fois en avril 2022 pour essayer de
voir ce que c'est et si elle a grandi.
