Après des semaines d'hospitalisation, Ashley, âgée de 34 ans seulement, a été
diagnostiquée comme souffrant d'une myopéricardite et ayant subi une crise
cardiaque due au vaccin Johnson \& Johnson. La myocardite et la péricardite sont
des inflammations du cœur ou du sac qui l'entoure.

Ashley nous raconte une partie de son histoire :

Trois mois plus tard, je suis ici, au 6e étage de l'unité de cardiologie. Je
dors avec des perfusions dans les deux bras, un moniteur cardiaque 24 heures sur
24, des ECG quotidiens, des radiographies de la poitrine, des analyses de sang
deux fois par jour, un échocardiogramme, un scanner cardiaque, une procédure de
cathétérisme cardiaque et, enfin, une IRM cardiaque pour montrer l'inflammation
et la cicatrisation de mon cœur..... indiquant une crise cardiaque et une
déchirure de l'artère coronaire ! Et maintenant, j'ai une incision dans l'aine
qui est très inconfortable !

Ils ont trouvé un bon cocktail de médicaments qui m'a aidé, surtout lorsque le
taux de troponine atteint 7,30 ! (0.4 - 0.39).

Je suis à la maison maintenant mais je ne peux pas soulever plus de 5 kilos, je
suis sous 6 nouveaux médicaments différents. J'ai toujours des palpitations
bizarres et aléatoires tout au long de la journée et de la nuit.

Le cardiologue dit qu'il faut 6 à 8 mois pour que la déchirure de l'artère
coronaire guérisse. Cependant, la myocardite peut revenir n'importe quel
jour/mois/année de façon inattendue.

Deux cardiologues et un médecin spécialiste des maladies infectieuses m'ont
demandé de ne JAMAIS recevoir un autre vaccin COVID !
