\testimony{Jaide  Marshall-Inman}
{Wagga Wagga, Nouvelle Gales du Sud, Australie}
{21 ans}
{AstraZeneca}
{20 août 2021}
{tremblements constants, douleurs musculaires, douleurs articulaires, maux de
  tête, douleurs abdominales, nausées, vomissements, sensibilité, fatigue,
  frissons chauds et froids, douleurs thoraciques sévères, essoufflement,
  caillots de sang aux poumons}
{picture.jpg}
{https://nomoresilence.world/astra-zeneca/jaide-marshall-inman-aged-21-astrazeneca-severe-adverse-reaction/}
{

Jaide raconte son histoire.

Je suis victime d'un effet indésirable du vaccin AstraZeneca, première dose, et
j'ai été hospitalisée 3 fois maintenant.

1ère fois: Je suis sortie de l'hôpital avec le diagnostic suivant “ce ne sont
que les effets secondaires du vaccin ”, “ une douleur musculaire dans la
poitrine ”, et on m'a renvoyée chez moi sans aucun suivi, juste des comprimés de
Panadol/Nurofen.

J'ai passé d'autres scanners ailleurs, sur demande de mon médecin généraliste,
et j'ai fini par avoir deux caillots de sang dans les poumons. J'ai été
maltraitée, rejetée, mal diagnostiquée et on m'a fait croire que ce n'était
qu'un sous-effet du vaccin.

Deuxième fois: Réadmission et traitement pour caillots sanguins, mise sous
anticoagulants pendant les 12 mois suivants.

3ème fois: je suis retourné à l'hôpital, j'avais toujours du mal à respirer et
des douleurs à la poitrine. On m'a renvoyée chez moi en me disant qu'on ne
pouvait plus rien faire pour moi.

J'ai eu des effets secondaires comme des tremblements constants, des douleurs
musculaires, des douleurs articulaires, des maux de tête, des douleurs
abdominales, des nausées, des vomissements, de la sensibilité, de la fatigue,
des frissons chauds et froids, des douleurs thoraciques sévères et un
essoufflement tel que je me suis évanouie avant d'entrer dans l'hôpital.

Personne ne m'a écoutée. J'avais du mal à respirer depuis que j'avais reçu le
vaccin. Ils ne comprennent pas ce que c'est que de lutter pour respirer à l'âge
de 21 ans.

Je veux que mon histoire soit entendue.

}
