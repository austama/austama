Alexa a reçu sa 1ère dose de Pfizer le 19 mars 2021. Une semaine plus tard, elle
était incapable de marcher sans se tenir à des objets environnants, ou de rester
debout pendant plus de 30 secondes sans perdre connaissance. Elle souffrait
surtout de vertiges invalidants et de migraines intenses. Elle doit maintenant
utiliser des appareils fonctionnels tels qu'une marchette, une canne et un
déambulateur.

De plus, elle souffre maintenant de douleurs articulaires, d'atrophie, de
faiblesse musculaire ainsi que des muscles gelés, d'une parathésie extrême du
côté gauche de son corps (côté injection), de douleurs nerveuses et de “
décharges ” électriques, de troubles de la déglutition, de changements dans son
cycle menstruel, d'acouphènes, d'une sensibilité extrême à la lumière et au son,
d'un manque de perception de la profondeur, de pertes de mémoire, d'une
diminution des capacités cognitives et d'une sensibilité intense aux
aliments. Elle a maintenant une mobilité limitée.

Son équipe médicale pense que ces symptômes sont dus à une inflammation
systémique liée au vaccin. Elle a été admise à l'hôpital pendant quelques jours,
et bien que de nombreux facteurs comme le SGB, la SEP et les troubles
auto-immuns aient été écartés, elle cherche toujours des réponses.

