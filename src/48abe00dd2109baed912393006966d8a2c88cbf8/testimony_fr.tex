\testimony{Julie  Chapman}
{Non communiqué}
{Non communiqué}
{AstraZeneca}
{25 février 2021}
{Symptômes neurologiques, picotements, douleurs nerveuses brûlantes, allodynie}
{picture.jpg}
{https://nomoresilence.world/astra-zeneca/julie-chapman-astrazeneca-severe-adverse-reactions/}
{

\normalsize

L'histoire de Julie nous est racontée avec ses propres mots:

Ma décision de prendre le vaccin Astrazeneca a eu un impact négatif sur nos
vies. J'étais auparavant une femme au sommet de sa forme physique. Je n'avais
aucun antécédent de maladie neurologique, ni aucune autre maladie grave. Je
faisais de l'exercice tous les jours, marchant pendant 3 heures par tous les
temps, je faisais du yoga depuis 10 ans et j'allais à la salle de sport.

J'ai su deux jours après le vaccin que quelque chose n'allait pas du tout
lorsque je me suis réveillée avec des picotements douloureux et aigus se
déplaçant rapidement dans tout mon corps. Je me sentais très mal, avec une forte
fièvre, une perte d'appétit et des douleurs depuis que j'avais été vacciné. Mais
ce symptôme était d'un autre niveau.

Il ne s'est pas calmé. Les jours se sont transformés en semaines. Les semaines
se sont transformées en mois. Au cours de cette période, j'ai contacté les
médecins généralistes et le 111, et même le 999 (Royaume-Uni) une nuit où une
crise de formication (sensation de particules solides se déplaçant sous la peau)
particulièrement grave m'a fait craindre pour ma vie.

Les symptômes sont apparus et ont disparu au fil des mois, comme des douleurs
nerveuses brûlantes dans tout le corps, ce qui m'a conduit à me rendre aux
urgences à plusieurs reprises, sans résultat positif. Pendant les trois premiers
mois, j'ai à peine dormi une heure en raison de douleurs nerveuses sévères sur
toutes les parties de mon corps sur lesquelles je posais mon poids (je ne
pouvais même pas dormir assise car le côté de ma tête me faisait mal). Je
pouvais encore bouger, alors certaines nuits, je marchais péniblement dans les
rues, seule, au lieu de rester douloureusement éveillée dans mon lit.

Les acouphènes me rendaient folle. La douleur dans mes yeux me donnait
l'impression de devenir aveugle. Tout sommeil se terminait brusquement par des
cris de douleur, de peur de devenir aveugle ou des deux. Les deux premiers mois,
je ne pouvais pas supporter les vêtements contre ma peau ou les draps de lit,
tant les nerfs de ma peau sont hyperactifs.

Bien que j'aie fait tous mes efforts habituels pour avoir une alimentation
saine, prendre des compléments alimentaires et faire de l'exercice, une maladie
mentale s'est abattue sur moi à cause de la peur des symptômes interminables et
de l'épuisement, ce qui a entraîné une anxiété et une dépression graves. Bien
sûr, je n'étais pas en état d'aller travailler et mon mari, épuisé, est devenu
mon soignant officiel à plein temps.

Cinq mois plus tard, certains symptômes se sont un peu atténués. La douleur
nerveuse brûlante s'est réduite à des picotements et des aiguilles. Elle est
encore assez constante et forte par moments (ce qu'on appelle en médecine la
dysesthésie). J'ai toujours des nerfs hyper sensibles dans ma peau, ce qui
m'empêche de porter des vêtements confortablement. J'adorais m'habiller, mais ce
plaisir a disparu. Le plaisir de prendre des bains de soleil a également
disparu. Le soleil aggrave la douleur nerveuse, elle m'affecte même à travers
les vêtements, donc marcher par une journée ensoleillée n'est plus un
plaisir. Les vacances ne seront pas un plaisir et même dormir dans un lit
étranger est impossible. Il faut que j'emporte ma propre literie
extra-douce. J'ai la sensation d'avoir pris un coup de soleil sous la peau
lorsque quelque chose me touche.

J'ai repris le travail, mais j'ai du mal à supporter la douleur, j'ai besoin
d'une housse pour ma chaise de bureau et je n'ai plus le plaisir de me rendre au
travail à pied par une journée ensoleillée. Je ne me sens plus la même
personne. Je souffre toujours de dépression et j'ai du mal à accepter ce qui
s'est passé.

Nous ne savons pas ce que l'avenir nous réserve. J'ai recours à l'homéopathie et
à la biorésonance, ce qui améliore mon état de santé général.

Il est encore trop tôt pour dire si cela aide les symptômes neurologiques. Un
neurologue m'a vu et il pense qu'il s'agit probablement d'un trouble
neurologique fonctionnel. Il m'a vu en mars et avait bon espoir que cela
disparaisse. Mais cela a dépassé le délai qu'il pensait. J'ai passé une IRM
normale et j'attends une deuxième IRM et des tests de conduction nerveuse.

Je voulais partager mon histoire pour sensibiliser le public.

}
