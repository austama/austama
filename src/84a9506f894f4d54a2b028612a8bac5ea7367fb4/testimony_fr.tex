J'ai passé deux jours à l'hôpital avec mon fils, les 19 et 20 mai 2021.NOTE : le
fils de Leigh-Ann, âgé de 18 ans, a également souffert d'embolies pulmonaires à
cause du vaccin de Johnson \& Johnson ! On lui a diagnostiqué des embolies
pulmonaires bilatérales et un infarctus du poumon gauche dus au vaccin Johnson \&
Johnson.

Six jours plus tard, le 26 mai, l'arrière de ma jambe gauche était si douloureux
que je ne pouvais pas marcher dessus. Mon fils m'a déposée aux urgences où l'on
m'a fait une prise de sang et mon Dimer D était de 101630 (la normale est de
0-500). On m'a fait passer une échographie Doppler et on a découvert une TVP
(caillot de sang) de 2,5 pieds de long dans ma jambe gauche. On m'a emmené en
ambulance dans un autre hôpital pour le faire enlever chirurgicalement.

Ils ont fait un scanner de contraste à l'hôpital suivant et ont trouvé des
embolies pulmonaires. Ce qui a empêché l'opération. J'ai passé une semaine à
l'hôpital sous perfusion d'héparine et 7 jours après sous injections de
tinzaperin. Je suis maintenant sous anticoagulants à vie.

Je suis retournée à l'hôpital une semaine plus tard avec une infection massive
des tissus mous dans mon bras, mon aisselle, ma poitrine et mon côté. Mes
symptômes ont commencé 19 jours après la vaccination par Pfizer, mais j'étais
tellement distraite par mon fils que je n'ai été traitée qu'une semaine plus
tard.

Depuis, j'ai été hospitalisée deux fois pour des hémorragies menstruelles et je
dois maintenant subir une hystérectomie éventuelle. J'ai un échocardiogramme, un
moniteur holter, un rendez-vous en cardiologie, un rendez-vous en chirurgie
vasculaire, un rendez-vous en hématologie et j'ai été envoyée en urologie pour
un problème rénal.

Je vis chaque jour brisée, tant physiquement que mentalement.

Personne dans les hôpitaux ne se soucie de ce qui vous arrive si vous n'avez pas
Covid. Pas un seul des médecins qui m'ont traité à l'hôpital n'a signalé ma
blessure !

J'ai depuis signalé mon cas au ministère de la Santé de la Saskatchewan, où un
médecin local a refusé ma demande de remboursement en raison de la blessure de
mon fils et d'un `` trouble sanguin antérieur non diagnostiqué ``. Mon
hématologue envisage de faire appel de cette décision. J'ai fait un rapport à
Pfizer, qui a l'obligation de faire un rapport à Santé Canada. Aucune
réponse. Je ne peux pas faire de demande d'indemnisation en raison du refus de
l'OHM de ma demande initiale en juin.

NOTE : le fils de Leigh-Ann, âgé de 18 ans, a également souffert d'embolies
pulmonaires à cause du vaccin de Johnson \& Johnson !
