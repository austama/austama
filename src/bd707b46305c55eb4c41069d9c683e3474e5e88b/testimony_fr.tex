Il y a un mois et demi, j'ai pris la première dose du vaccin Pfizer et 5 jours
après, j'ai commencé à avoir une vision floue dans mon œil gauche. En 72 heures,
j'ai perdu 70\% de ma vision dans l'œil gauche. Deux optométristes et deux
ophtalmologues ont diagnostiqué une rétinopathie séreuse centrale (RSC) - un cas
de décollement de la rétine. Trois d'entre eux affirment “officieusement” la
possibilité d'un lien entre ce problème et la dose Pfizer que j'ai reçue, l'un
d'eux le nie fermement.

Mon problème actuel est que, bien qu'ils disent qu'il n'y a aucune garantie que
cela ne se reproduira pas ou n'empirera pas si je reçois d'autres doses, ils ne
peuvent pas m'exempter de les recevoir, car la cause légale valable pour
l'exemption est, si quelqu'un a des “effets secondaires graves”, ce qui est
défini comme “seulement en cas d'hospitalisation”, et perdre 70\% de votre vue
n'est pas considéré comme un effet secondaire grave !

Je travaille sur des ordinateurs pour gagner ma vie, donc je ne peux pas
travailler, je ne peux pas être embauché dans le commerce de détail, je ne peux
même pas quitter le pays en avion !

Je suis traumatisée. J'ai des crises de panique et je fais des cauchemars où je
deviens complètement aveugle.
