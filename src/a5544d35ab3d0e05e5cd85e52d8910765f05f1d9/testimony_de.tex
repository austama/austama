Mathilde erzählt selbst:

Ich habe meine erste Dosis Pzifer am Dienstag, den 25. Mai erhalten. Am
Mittwochabend und Donnerstag hatte ich eine violette Färbung der linken Hand
(Impfstoff für den linken Arm) und eine Lähmung auf dieser Seite. Ich konnte
meinen Arm nicht bewegen und fühlte, dass mein Arm hart und schwer war.

Am nächsten Tag waren die Armschmerzen weg, aber mein linkes Bein begann zu
schmerzen. Die Schmerzen waren erträglich, aber ich hinkte. Am Montagmorgen
stand ich auf und konnte die Ferse nicht mehr absetzen, es war super
schmerzhaft. Ich konnte meine Zehen nicht mehr bewegen, sie waren
eingefroren. Ich wurde ins Krankenhaus gebracht und eine Woche lang mit Morphium
und Ketoprofen behandelt, aber nichts half. Ich war 10 Tage lang im
Krankenhaus. Es wurde eine lumbale und medulläre Gehirnuntersuchung und ein EMG
durchgeführt und eine Radikulitis bestätigt.

Sie versuchten, die Morphiumdosis zu erhöhen und dann Kortikosteroide zu
verabreichen, aber nichts lindert die Symptome. Ich leide darunter, dass ich
nicht richtig sitzen, liegen oder stehen kanm, oder beim. Laufen hinke - ich
habe keine Stabilität.

Ich kam aus dem Krankenhaus, nachdem ich 5 Tage Immunglobuline und eine
Lumbalpunktion erhalten habe.

Ich bekam eine Woche später zu Hause schreckliche Nebenwirkungen: Erbrechen,
starke Kopfschmerzen, Müdigkeit. Ich konnte nicht laufen.

Wieder wurde ich eine Woche später ins Krankenhaus eingeliefert, um ein EMG und
eine Immunglobulinbehandlung zu wiederholen. Ich hatte große Blutverluste
(Gerinnsel). Wieder versuchten wir es mit Acupan und Tramadol... dann ein
zweites EMG, das bestätigte, dass das Radikulit immer noch vorhanden ist.

Heute, 2 Monate nach der Injektion, nehme ich Antiepileptika, Neuroleptika,
Ketoprofen und Triptan 2 x täglich, weil meine Kopfschmerzen so heftig und
schwer zu ertragen sind.

Ich weiss nicht, wann ich genesen werde, da es kaum bekannte Fälle von
Radikulitis nach der Impfung gibt.

Ich habe mich impfen lassen, um eine Stelle im Gesundheits- und Sozialwesen zu
bekommen, und heute muss ich die Stelle aufgeben. Ich werde nicht mehr arbeiten
können, da mein Vertrag in 3 Wochen ausläuft. Ich bin wütend und angewidert!

Auch La Presse de Gray hat über Mathildes Fall berichtet:

``Als Mathilde Harismendy einen befristeten Vertrag bekam, riet ihr Vorgesetzter
ihr ``dringend'', sich impfen zu lassen, um ihren Arbeitsplatz zu behalten, was
sie auch tat. Das Ergebnis ist, dass ihr Alltag heute zur Hölle geworden ist!
Die Zeiten, in denen Mathilde Harismendy einkaufen ging, mit ihren Kindern am
See spazieren ging oder Rad fahren konnte, sind nun vorbei. Allein einkaufen
gehen oder einfach ihre Eltern besuchen, diese Aktivitäten, die vielen Menschen
normal erscheinen, sind für die 27-jährige Frotéenne, die seit Mai gegen ihren
Willen von ihren alltäglichen Tätigkeiten Abschied nehmen musste und nun auch
ihrem Job verloren hat, zur Tortur geworden.''
