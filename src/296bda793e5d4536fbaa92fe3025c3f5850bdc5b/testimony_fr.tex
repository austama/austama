Michelle, du Royaume Uni, a été l’une des premières de moins de 40 ans à être
contactée et à qui l’on a offert le Vaccin AstraZeneca en raison de son cancer
du sein diagnostiqué 10 ans auparavant. À cette époque elle travaillait dans une
école et a accepté la vaccination afin de protéger son entourage. Après sa
vaccination elle a développé un gonflement et une faiblesse dans ses jambes,
bras, dos et sa mobilité s’est détériorée au point que son travail
essentiellement l’a fait partir.

Agile jusqu’à maintenant, elle ne peut quitter la maison et ne peut marcher sans
(une) aide. Malgré 5 transports en ambulance à l’hôpital et rendez-vous chez le
rhumatologue, personne ne veut reconnaître que le Vaccin AstraZeneca a un rôle à
jouer dans ce qu’ils ont diagnostiqué maintenant comme un M.E! Après 8 mois elle
n’a toujours pas de retour de son cardiologue qui lui a dit qu’il chercherait à
la mettre dans la liste des urgences pour être reçue par un neurologue !
Michelle explique plus moi dans sa vidéo…

