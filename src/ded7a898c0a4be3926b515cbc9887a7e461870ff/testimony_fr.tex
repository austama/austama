La communauté médicale taïwanaise est sous le choc après le décès du Dr Vincent
Wang, qui était l'ancien directeur du département des maladies infectieuses de
l'hôpital Mackay Memorial à Taipei.

Pfizer avait envoyé une lettre l'invitant à prendre la dose “booster” de leur
vaccin.

Erica raconte la perte de son mari suite à la prise de la 3ème dose (dose de
“rappel”) Pfizer:

“Mon mari était un médecin spécialisé dans les maladies infectieuses. Après
avoir reçu sa troisième injection Pfizer, il a eu des vertiges, une baisse de
tension et s'est évanoui dans les escaliers le 5 septembre. Mon mari est décédé
au milieu du mois dernier à cause du vaccin anti Covid-19.

Son médecin ne lui a pas donné le temps d'aller à l'hôpital. Il se sentait mieux
le 15 septembre, mais est décédé le 16 septembre tôt le matin, durant son
sommeil profond.

Nous avons été très occupé à devoir appeler le système médical complexe aux
Etats-Unis afin d'obtenir ses dossiers médicaux, et nous avons également
complété notre propre anatomie pathologique (le non-suicide et le crime doivent
être à nos frais), mais les résultats du rapport arriverons dans six à sept
semaines. Cela fait presque un mois, et nous n'avons toujours pas reçu de
certificat de décès, donc il ne peut toujours pas être incinéré.

Nous avions aussi besoin d'une autopsie privée et nous avons fait un rapport au
CDC et à Pfizer. A l'heure actuelle, nous n'avons toujours pas reçu de
dédommagement.

Il n'a reçu aucun traitement de la part de la communauté médicale… son médecin
lui a dit de boire plus d'eau!

Nous avons perdu le père de mes deux enfants, notre support financier, et j'ai
perdu l'amour de ma vie. Donc, son départ nous a gravement atteint et nous le
pleurons encore.

Nous sommes parti de Taïwan et arrivé aux Etats Unis il y a deux ans et demi
durant l'été. Nous n'avions aucun contact là-bas, et il nous avait fallu
beaucoup de courage pour arriver en Amérique à l'âge de cinquante ans.”

