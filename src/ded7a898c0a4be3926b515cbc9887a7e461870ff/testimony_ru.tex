едицинское сообщество Тайвани находится в шоке после смерти доктора Винсента
Ванга, бывшего директора инфекционного отделения больницы Mackay Memorial в
Тайбэе.

Компания Pfizer отправила ему письмо с предложением принять «бустерную» дозу их
вакцины.

Эрика вспоминает о потере мужа после приема третьей дозы («бустерной» дозы)
Pfizer:

«Мой муж был врачом-инфекционистом. После третьей инъекции Pfizer он
почувствовал головокружение, у него снизилось кровяное давление и 5 сентября он
потерял сознание на лестнице. Мой муж умер в середине прошлого месяца от вакцины
Covid-19.

Его врач не дал ему времени обратиться в больницу. 15 сентября ему стало лучше,
но 16 сентября он скончался рано утром в глубоком сне.

Мы были очень заняты нашим обращением в сложную медицинскую систему в
Соединенных Штатах, чтобы получить разрешение на составление медицинского досье,
мы заполнили его нашими данными (нужно было отметить, что здесь нет суицида или
убийства, иначе это было бы за наш счет). Результаты освидетельствования придут
через шесть-семь недель. Прошел почти месяц, а свидетельство о смерти до сих пор
не получено, поэтому мы не можем его кремировать.

Нам также потребовалось частное вскрытие, о котором мы сообщили в CDC и
Pfizer. На данный момент мы всё ещё не получили компенсацию.

Он не получил никакого лечения от медицинского сообщества… его врач сказал ему
пить больше воды!

Мы потеряли отца двоих детей, нашу финансовую поддержку, а я потеряла любовь
всей моей жизни. Его уход очень тяжело повлиял на нас, и мы до сих пор
оплакиваем его.

Мы покинули Тайвань и приехали в Соединенные Штаты два с половиной года назад
летом. У нас здесь не было контактов тогда, и нам потребовалось немало мужества,
чтобы переехать в Америку в возрасте пятидесяти лет».
