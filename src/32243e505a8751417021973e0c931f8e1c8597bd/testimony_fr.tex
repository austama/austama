Katy, la femme de Charles, raconte son histoire:

Charles allait bien le matin et était partit chercher de quoi manger. Il me dit
qu'il se sentait un peu “déconnecté” et me demandé si je pouvais conduire. J'ai
du conduire moins d'un kilomètre quand il sursauta (mon mari est un homme et
rien ne m'effraya). Je lui ai demandé si tout allait bien et elle me dit: “Ça
m'a effrayé!”, je lui ai demandé “qu'est-ce qui t'as effrayé?” et il me répondit
“il y a juste trop de chose à comprendre.” cela me préoccupa mais je continuais
de conduire vu que nous arrivions au restaurant. Il sursauta ne nouvelle fois et
j'ai pu comprendre que quelque chose n'allait pas du tout.

Je suis remonté dans la voiture et alla directement aux urgences qui étaient au
niveau du même parking. J'ai couru jusqu'à son coté de la voiture et essayé de
le faire sortir, mais il était en train d'avoir des hallucinations et ne me
répondait pas, vu qu'il était préoccupé avec les hallucinations. J'ai essayé de
le tirer de la voiture et j'ai finis par réussir à le tirer sur le sol entre la
porte et le terre-plein central couvert de paille sur une des places de parking.

J'ai couru à l'intérieur et hurlé à l'aide. Deux infirmières sortir
immédiatement avec moi et essayèrent de le tirer du terre-plein. Une ou deux
minutes passèrent jusqu'à ce qu'un infirmier arriva et fut capable de le mettre
sur le terre-plein où ils purent travailler sur lui.

À ce moment, il n'était pas vraiment lucide. J'étais hors de moi, mais je ne
suis pas du genre à paniquer, donc j'ai essayé de rester le plus calme
possible. Ils essayèrent de l'évaluer, mais ne savait pas ce qui n'allait
pas. J'étais au téléphone en train d'essayer d'appeler un ami pour de l'aide.

Les infirmières commençèrent à dire “lèves toi, Charles!”, j'ai accouru et
essayé de le relever aussi. Il était en train de baver et ronfler. Quelque chose
au fond de moi me disait que ça n'allait pas le faire. Les urgences avaient
apparamment appelé le 911 [ndt: numéro d'urgence], parce qu'une ambulance était
en train d'arriver. Ils commençèrent par essayer de le “stabiliser” et firent un
CPR [ndt: à voir] et d'autres choses. À ce moment, je me souviens clairement
l'avoir regarder sur le terre-plein et je disais “il semble mort” j'étais
dévasté!!!

Les aide-soignants continuèrent de travailler et me dirent qu'ils devaient
l'emmener à l'hopital une fois que son état serait stabilisé. Quelques minutes
plus tard, un des aide-soignants vint me voir et me dit: “Je veux vous dire que
je suis vraiment fier de vous! Vous avez tout ce qu'il fallait, l'emmener
jusqu'ici et tout le reste.” Je savais ce qu'il était en train de se passer, et
ils ne voulaient pas que je me sente coupable.

Je leur ai demandé comment il allait, et ils me dirent qu'ils étaient toujours
en train d'essayer de le réanimer - ce qui me donna un faux espoir. Finallement,
ils me dirent qu'ils l'emmenait en ER [ndt: à changer], et une nouvelle fois,
j'ai pensé qu'il y avait un peu d'espoir vu qu'ils avaient précédemment dit
qu'ils ne pouvaient pas l'emmener tant qu'il n'était pas stable. A ce moment,
nos amis proches, qui sont essentiellement la mère de mon mari, sa soeur et sa
nièce, arrivèrent et partir à l'ER [ndt].

Quand nous y sommes arrivé, nous étions escorté jusqu'à une salle d'attente et
on nous demande d'attendre. Un autre ami proche arriva. Nous nous sommes assis
et avons attendu plein d'espoir. Alors ils arrivèrent et nous annoncèrent qu'il
avait été placé en soin critique et qu'un docteur était avec lui. Encore plus de
faux espoirs.

À un moment, le docteur entra et demanda qui était sa femme. Je me suis présenté
et le docteur me demanda exactement ce qui c'était passé. Je lui raconta
l'histoire et je pensais qu'elle était en train de le diagnostiquer. Ce n'était
pas le cas. Elle venait m'annoncer qu'ils n'avaient pas réussi à le réanimer et
qu'il était décédé.

Nous nous sommes tous effondré en larme. L'assistant du médecin légiste arriva et nous demanda si nous voulions réaliser une autopsie. Son assistant dit qu'il verait si il pouvait en réaliser une, mais la plupart des comtés de GA partagaient les examinateurs, et que nous devions demandé une autopsie. Il me dit qu'il me tiendrait au courant.

Le jour suivant, l'assistant du médecin légiste appela mes amis (ils lui
donnèrent leur numéro de téléphone) et leur dit que mon mari avait souffert d'un
“évènement hypoxic.” Il dit qu'ils n'allaient pas faire d'autopsie, parce qu'il
savait ce qu'il l'avait tué. c'était ça. Quand j'ai reçu le certificat de décès
de l'hopital, le document disait que la cause du décès était une “maladie
cardiovasculaire hypertensive.” je n'ai jamais eu une réponse clair sur ce qui
tua mon mari.

Mon mari était mon monde. Il décéda 2 jours avant notre 30ème annivaire se
mariage, et je ne serais plus jamais la même. L'avoir perdu, et devenir veuve à
51 ans m'a dévasté. Je regrete qu'il ai pris le vaccin parce que je sais au fond
de mon coeur que ça à au moins été un facteur, et ce qui l'a probablement tué.

Il allait parfaitement bien avant cet évènement. Il était un homme
incroyablement formt - littéalement, les gens l'appelait “le tremplement de
terre” parce qu'il était tellement fort - et je ne peux pas accepté que ce soit
une genre de maladie cardiovasculaire, surtout quand il n'a jamais montré aucun
signe de problème cardiaque.

J'ai l'impression que mon avenir a été dissous devant mes yeux. J'ai perdu mon
meilleur ami, et mon rocher. Je n'ai pas travaillé pendant 3 mois. Je n'ai
jamais été sans mon mari depuis que j'ai 19 ans et je ne suis jamais vraiment
devenu une adulte sans lui.

Je suis en colère, mais pas contre Dieu… Je n'ai pas vraiment d'endroit où
diriger ma colère. Je suis brisée. Je suis terrifiée par l'obligation d'être
vacciné parce que je sais pourquoi je suis devenue veuve à 51ans. Je ne sais pas
quoi dire d'autre à part que je suis brisé, seule et dévastée.

