История 19-летней Шандриль рассказывается ее собственными словами:

Меня зовут Шандриль, мне 19 лет. 22 июня мне сделали прививку первой дозой
вакцины Pfizer, и через неделю у меня начались непрекращающиеся головные боли.

2 июля я обратилась в больницу, и после томографии врач сказал мне, что у меня
церебральный тромбофлебит (редкая форма инсульта), и меня госпитализировали на
10 дней.

На следующий день после выписки из больницы у меня начались проблемы со
зрением. Тогда, 14 июля я вернулась в отделение скорой помощи, и после
нескольких обследований врач сказал мне, что у меня осложнение, связанное с
инсультом. Офтальмолог объяснил мне, что у меня было слишком большое давление в
черепе, что привело к отеку диска зрительного нерва и «временному отключению»
зрительного нерва в моём левом глазу.

22 июля невролог сделал мне люмбальную пункцию, чтобы снизить это давление. Я
всё ещё на таблетках для снижения давления, а также на препаратах для разжижения
крови.

После пункции мое зрение улучшилось, но ещё не полностью восстановилось. У меня
всё ещё запланировано много обследований, и должно пройти ещё несколько месяцев,
прежде чем я надеюсь вернуться к 100\% зрению.

Все мои тесты для определения проблемы вызвавшей свертываемость крови
отрицательные. Таким образом, единственным правдоподобным объяснением моего
заболевания остается иммунизация вакциной!
