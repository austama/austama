История Энджи, рассказанная ее дочерью:

Моя мама чувствовала необходимость вакцинации, потому что три года назад у нее
была бронхиальная пневмония, и она не хотела снова оказаться в такой же
ситуации. Поэтому, в конце января 2021 года, она получила вакцину. Вскоре после
этого она начала страдать от временных ишемических атак (непродолжительная
закупорка артерии в головном мозге, что приводит к потере двигательной функции
конечностей и параличу лицевого нерва).

Телефонные звонки к врачу остались без ответа, и трижды она попадала в больницу,
чтобы опять вернуться домой. В начале марта у неё развилась гемиплегия
(массивный паралич) правой руки, всего через 8 часов после второй вакцины. Она
лежала одна, холодная, испуганная и изо всех сил пыталась встать. В течение 10
часов она пыталась дотянуться до телефона посреди ночи.

Атака ограничила её речь, питание, общение и подвижность. На протяжении всех
последних дней своей жизни она страдала от рецидивирующих инфекций легких и
мочевого пузыря. Она очень ослабла, потому что они отказывались давать ей пищу
или питье. Кормили зондом через нос. Я чувствовала себя беспомощной, потому что
разрешали только один визит в день и не более одного часа, и то, если повезёт.

Я просила о многом, чтобы сделать её жизнь более комфортной и безболезненной. Я
задавала много вопросов, но тщетно. Я старалась быть спокойной и дипломатичной
(хотела знать, хорошо ли с ней обращаются, когда меня не было рядом), но в конце
концов она умерла, задыхаясь от собственной мокроты: бронхи забились, а легкие
наполнились. Ей не дали никакого лечения: ни кислорода, ни антибиотиков, ни
ингаляции…. ничего !

Она была одна и напугана. Они солгали мне, сказав, что она немного заболела. Они
лишили её всякого достоинства. Я смотрела на мою мертвую маму, а они говорили
мне в это время, что она не умерла, но в забытьи! Я держала её за руку и
запястье, но пульса не было. Она была белая, как снег. В течение десяти минут
мне пришлось терпеть унижение медсестер, которые проводили странный ритуал по
проверке её пульса, прежде чем признать мою правоту.
