\testimony{Bruno Oscar Graf}
          {Santa Catarina, Brésil}
          {28 ans}
          {Astrazeneca}
          {14 août 2021}
          {AVC,  Thrombocytopénie prothrombotique immunitaire}
          {picture.jpg}
          {https://nomoresilence.world/astra-zeneca/bruno-oscar-graf-died-from-astra-zeneca-vaccine/}
          {

L'avocat et résident de Blumenau (qui est une municipalité de l'Etat de Santa
Catarina), Bruno Oscar Graf, est mort à 28 ans le Jeudi 26 aout suite à un AVC
résultant d'une thrombose. La mère du jeune homme, Arlene Ferrari Graf, pense
que le vaccin AstraZeneca, a pu causer la mort de son fils. Il s'est fait
vacciner le 14 aout, 12 jours avant sa mort, au Centre de Vaccination localisé à
Parque Vila Germanica.

Arlene dit que Bruno a ressenti de puissants maux de tête et de la fièvre, et
que c'est ce pourquoi il a été hospitalisé le Lundi 23. Mardi, il aurait subi un
accident vasculaire cérébral (AVC) hémorragique, qui a entraîné plusieurs
complications pour sa santé et a évolué vers la mort deux jours plus tard.

“Il a fait une prise de sang qui montrait un taux de plaquettes bas et un haut
taux de Protéine C-réactive. Ils ont suspecté le covid ou la dengue. Ils n'ont
rien fait de plus que de lui donner des médicaments pour soulager ses maux de
tête. (…) Le Mardi, à 7 heures, il a fait un AVC et il n'y avait plus rien à
faire”, a dit sa mère.

Sur les réseaux sociaux, Arlene a fait une publication rapportant la mort de son
fils:

“Bruno était un fils parfait dans tous les sens possibles à imaginable. Un
enfant aimant, attentionné, aidant et bon. Dieu, dans son infinie miséricorde,
nous a béni de ces presque 29 ans de coexistance heureuse, où l'amour a
reigné. Pas un seul jour n'est passé sans que je ne l'embrasse et lui dise:
“Fils, ta mère t'aime. - Je t'aime aussi, maman.”… J'aime mes enfants, et
l'affection fait partie de nos vies quotidiennes”, a-t-elle écrit dans le post.

Dans un autre extrait, elle explique que le père de Bruno a décidé de donner ses
organes, pour que “d'autres parents puissent sourire, et que le coeur de Bruno
puisse continuer à battre”.

Arlene a commenté que le certificat de décès stipule que l'AVC était dû à une
Thrombocytopénie prothrombotique immunitaire. Après avoir fait ses recherches,
elle a appris que le vaccin pouvait être la cause de sa thrombose.

De plus, elle a abordé le sujet avec les docteurs qui s'occupaient de Bruno,
certains ont confirmé qu'il y avait un lien avec le vaccin et, d'autres, par
peur, avaient simplement hoché la tête lorsqu'elle leur posait la question.

Arlene a également rapporté qu'un examen pour investiger sur les liens avec le
vaccin et la thrombose avait déjà été entamé et qu'ils attendaient désormais les
résultats.

“J'attends les résultats d'un test appelé anti-héparine PF4, qui est allé
jusqu'en Europe, ça devrait nous donner des réponses. Mais je me base sur
l'ensemble des facteurs qui ont entraîné la mort de mon fils bien-aimé”,
a-t-elle déclaré.

L'Hopital Santa Catarina, le lieu de la mort de Bruno, n'a pas voulu commenter
l'affaire devant la presse car ils sont toujours entrain d'attendre les
résultats de l'investigation.

Bruno Graf avait été obligé de se faire vacciner, puisqu'il devait déménager en
Europe, où il s'apprêtait à se marier et à fonder une famille.

Comme d'habitude, les médias mainstream ont ignoré l'affaire, le gouvernement
rapportant que les bénéfices valaient plus que les risques, et qu'il est trop
tôt pour tirer des conclusions. L'examen pour prouver que la mort a été
entrainée par le vaccin a dû être payée par la famille, et a coûté environ 800
dollars, soit plus de trois salaires minimum au Brésil.

}
