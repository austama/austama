Mon compagnon a reçu sa seconde dose d'AstraZeneca le 6 juillet 2021. Il était
en forme et en bonne santé juste avant, et ne prenait pas d'autre médicament.

Il a cru être malade et avoir la diarrhée et est rentré du travail le 16
juillet. Il ne se sentait toujours pas bien le 17, et cette nuit à environ 23
heures, il s'est effondré sur le sol de la cuisine et a fait une crise
clonico-tonique.

Après avoir appelé les ambulances, ils ont mis environ 2 heures et demi pour
arriver. Pendant ce temps, il a fait 4 autres crises d'épilepsie. Il en a fait
une dans la salle de bain, où j'ai pu constater qu'il n'avait pas la diarrhée,
mais qu'il s'agissait en fait d'une hémorragie de sang coagulé.

Il a été emmené à l'hôpital seul, à cause du covid. Je suis arrivée à l'hôpital
4 heures plus tard et on m'a dit qu'il avait une hémorragie dans l'estomac et du
cerveau.

Il allait a peu près mieux, mais après les IRM, scanner, coloscopie et
endoscopie, deux transfusions de sang et des plaquettes données, nous n'avons eu
aucune réponse et le personnel médical est dédaigneux lorsqu'on demande si ça
pourrait être en relation avec la vaccination.

Il a été très instable sur ses pieds et une faiblesse dans le bras et la jambe
gauche, avec une bosse qui est apparue pendant la nuit. “Il n'a pas eu
d'accident vasculaire cérébral”, nous a-t-on dit. Il prend de très fortes doses
de Keppra, ce qui le rend fatigué et très irritable.

Il ne peut pas travailler et ne reçoit qu'une indemnité de maladie légale. Je
suis actuellement en arrêt pour pouvoir veiller sur lui.

C'est écoeurant qu'ils ne veuillent pas nous donner de réponse directe. Je ne
dirai jamais à personne de ne pas se faire vacciner, mais je suis en colère que
personne n'admette qu'il y a un réel problème. Comment peuvent-ils avoir des
statistiques correctes s'ils ne reconnaissent même pas les personnes handicapées
ou mortes?

